\section{Process}

\subsection{Backups}
Backing-up a document should be a priority, not an afterthought. 
Rewriting a paragraph is unfortunate, rewriting an entire thesis would be soul crushing. 
Multiple backup methods exist such as \href{https://onedrive.live.com/login/}{OneDrive}, \href{https://www.dropbox.com}{DropBox}, \href{https://github.com/}{GitHub}, \href{https://www.overleaf.com/}{OverLeaf}, or physical local backups, to name a few. 
While one is good, it is suggested to use multiple backup methods should a device or service become unexpectedly unavailable. 
This backup method also applies to data, always ensure data is backed-up. 

\subsection{Revision Control}
Revision control refers to the process of tracking changes to a document, or structured information. 
When writing, revision control takes two forms, iterative and milestone revision control. 

Iterative revision control refers to tracking the small changes a document naturally undergoes. 
Maybe a paragraph was removed in error, instead of rewriting, proper revision control should allow the document to be rolled back to a state where that paragraph exists.
A very powerful program for this type of revision control is \href{https://github.com/}{GitHub}, which integrates with both physical machines and \href{https://www.overleaf.com/}{OverLeaf}.

Milestone revision control refers to when a document is in a state to be reviewed. 
As indicated by C. Johansen (personal communication, May 25, 2023), revisions should be denoted sequentially following R1, R2, R3. Documents with feedback will have the reviewer's initials appended to the filename. An example document filename on revision three that was reviewed by C. Johansen would appear as: \verb*|AIAA_Scitech2024_Abstract_RocketTests_Smith_R3_CJ.pdf|.