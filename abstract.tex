% PURPOSE: Each full-length paper must have a summary-type abstract of 100 to 200 (maximum) words in one paragraph, without numerical references, acronyms, or abbreviations. The abstract indicates the subjects dealt with in the paper and states the objectives of the investigation.
% SOURCE: https://www.aiaa.org/publications/journals/Journal-Author#follow-the-minimum-formatting-requirements
\begin{abstract}
	%The development of large-scale supersonic aircraft has always been challenging as numerous problems exist in control, aerodynamics, handling, propulsion, and structural design. 
	%The difficulty of these problems increases when designing small-scale supersonic aircraft, and their successful development has remained elusive. 
	The flying and handling qualities of a Small-scale Supersonic Uncrewed Aerial Vehicle (SSUAV) are analyzed to facilitate future SSUAV design and experimental testing. 
	For this goal, the flying qualities of an experimental Multipurpose Uncrewed Fixed-wing Advanced Supersonic Aircraft (MUFASA) SSUAV are assessed. 
	%Aerodynamic coefficient data is obtained, and a Newtonian mathematical model is created to facilitate the simulation and evaluation of the MUFASA SSUAV. 
	%The flying qualities of the MUFASA SSUAV are evaluated against existing crewed aircraft standards. 
	%Specifically, the Froude scaling method is used as it provides a way to quantitatively compare small-scale vehicles to existing full-scale vehicle standards. 
	%The results obtained show that the handling characteristics of the MUFASA SSUAV are acceptable at transonic cruise conditions. 
	A continuous handling quality evaluation is proposed and implemented across the SSUAV's flight regime, providing a new flight trajectory optimization method. 
	The results highlight that the mean handling qualities of the targeted SSUAV range from acceptable to controllable in the transonic flight regime and controllable in the subsonic regime. 
	Finally, SSUAVs were compared to small-scale UAVs and full-scale supersonic aircraft, exhibiting much higher roll-rates. 
	When attitude rates are normalized, SSUAVs exhibit roll behavior in line with small-scale UAVs but pitch behavior in line with full-scale supersonic aircraft. 
	SSUAVs pose unique handling quality challenges that combine elements of small-scale UAVs and large-scale supersonic aircraft. 
\end{abstract}