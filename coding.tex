\section{Conclusions} \label{sec:Conclusion}

This paper obtained SSUAV aerodynamic and flying quality data by modeling the MUFASA and GOJETT SSUAV concepts. 
The MUFASA SSUAV flying quality parameters were evaluated against a proposed continuous modification of the MIL-STD-1797A standard for full-scale aircraft via Froude scaling. 
MUFASA exhibited mostly acceptable (level 2) performance at the analyzed cruise conditions of 4km altitude and 350$\tfrac{m}{s}$. 
This evaluation was then performed across MUFASA's flight regime to generate a handling quality level surface, a new tool for flying quality based flight trajectory optimization. 
Though the proposed modification to MIL-STD-1797A was designed to yield a continuous handling quality evaluation, abrupt transitions were still present. 
These abrupt handling quality level transitions are attributed to slight aerodynamic coefficient changes occurring where data was collected and a lack of specifications to quantify instability. 


When comparing each aircraft at its cruise conditions, MUFASA and GOJETT achieved a roll-rate at least twice as fast as small-scale UAVs and an order of magnitude faster than full-scale supersonic aircraft. 
Regarding pitch-rate, MUFASA and GOJETTS's performance was in line with full-scale supersonic aircraft. 
When normalized, SSUAVs exhibit roll behavior in line with small-scale UAVs but pitch behavior in line with full-scale supersonic aircraft. 
Taken together, the MUFASA and GOJETT SSUAVs pose unique handling quality challenges that combine elements of small-scale UAVs and large-scale supersonic aircraft. 
