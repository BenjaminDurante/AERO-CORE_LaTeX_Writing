\section{Introduction}
\lettrine{S}{mall}-scale Supersonic Uncrewed Aerial Vehicles (SSUAVs) have the potential to revolutionize high-speed research and civil transportation. 
The next era in supersonic civil transportation is approaching, with multiple supersonic aircraft concepts proposed for development \cite{A248Sun2017}. 
Unfortunately, supersonic prototype aircraft are costly, and the financial risk of development to companies is large \cite{NewsAerionFailureForbes}. %,NewsAerionFailureCNBC}. 
An alternative to full-scale prototype aircraft development is an SSUAV. 
Small-scale UAVs are now increasingly used as low-cost technology testing platforms \cite{A319Sobron2021}, and developing an SSUAV would significantly reduce the costs of supersonic aircraft development \cite{A244Eckstrom1975}. 
A handful of developmental research programs have investigated the feasibility of an SSUAV demonstrator \cite{mizobata2005development, A297Barbosa2014, A19Walter2012, A256Jacob2021, MachInitiative,A381Durante2022, A355Nelson2022}, however, only six programs (Ohwashi, MUFASA, SCALOS, R-UAV, Project Boom, and The Mach Initiative) are ongoing according to recent publications \cite{A381Durante2022, A256Jacob2021, A355Nelson2022, MachInitiative,newOowashi}. 
Of the aforementioned projects, none have undergone high-speed flight testing, and the fastest small-scale aerial vehicle flown remains the Trance remote piloted UAV \cite{Trance_2017}. 
One reason for the lack of a functioning SSUAV is that the aerodynamic conditions experienced by an SSUAV and the control required remain undetermined \cite{A355Nelson2022}. 
Whether SSUAVs pose unique handling quality challenges is an open research question that requires further study. 
Different SSUAV control strategies have been implemented \cite{A15Burnashev2019a, A103Langston2016, A279UEBA2021, D20Wienke2011}; however, due to the absence of standardized performance criteria, none of the control laws can be deemed reliably satisfactory \cite{Text2015AircraftControlAndSimulation}. 
Unfortunately, fixed-wing UAV flying quality performance standards are nonexistent \cite{A304Klyde2020}, with UAV flying quality research still in its infancy \cite{D37Cotting2010, A304Klyde2020}. 
To facilitate UAV flying quality standard development, flying quality data from multiple aircraft types must be generated \cite{A390Greene2014, A387Holmberg2008}. 
To facilitate the inclusion of SSUAVs in future UAV flying quality standards, SSUAV flying quality and perturbation response data must be generated.